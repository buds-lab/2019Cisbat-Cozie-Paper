% !TEX root = 99_main.tex

\subsection{Large uncontrolled data vs. small controlled data}

%Let us take a step back and observe how market analysis was conducted. Traditionally, super markets would call individual households and ask for their feedback about what products they would like to have in stock. In modern times, applications such as facebook mine preference data from users based on what posts they "liked", and use this to generate targeted advertising for each individual.

%One can draw these parallels to human comfort surveying. 

Placing a group of participants in a controlled experimental space and conducting feedback surveys is a trusted traditional method for human comfort surveys. Giving each participant a smartwatch, and analysing the patterns of hundreds of data points per user would be more akin to modern data analytics employed in industries outside the building sector. Both methods are useful for reaching different types of conclusions. The traditional method can derive conclusions such as: \emph{4 of the 15 users felt warm at temperatures higher than 25.6 $^\circ$C}. This insight is useful in the context of the previously mentioned generalizable thermal comfort models that are traditionally created in built environment research, but have poor accuracy. On the other hand, the uncontrolled, large data method can draw conclusions such as: \emph{4 of 15 users can be categorized as a user type that prefers cooler working environments}. This paper focuses on the use of clustering to show the groups of \emph{comfort personality types} that correspond with different behaviour in giving feedback and interacting with the spaces. 


%To further add to this, the uncontrolled method has minimal management overhead, and can be scaled by purchasing more devices, thus providing an even richer data set along the user axis. 

\subsection{In-situ benefits and limitations}
Uncontrolled experiments have minimal management overhead, which means that it can be easily scaled to larger groups by purchasing more devices. Furthermore, the users are analysed in their natural work environments and give feedback with a simple click on their watch. This reduction of effort results in no fatigue in the number of voluntary responses given as shown in Figure \ref{fig:summary}c. While users generally work from their office, they sometimes work from home, or at a local cafe. This presents a limitation in the context of traditional small controlled data. Large enough data needs to be obtained to filter out these scenarios and interpret meaningful results.


%The cozie watch face enables users to be analysed in-situ. By this we mean that the users work in their natural work environment, and give feedback with minimal effort. This allows the experiments to be conducted and scaled with minimal management overhead and there is no observable drop in voluntary responses in Figure \ref{fig:responseRate}. 

%there is no observable decrease in the feedback given. In fact, some of the users have enjoyed owning a fitbit, and will keep the device provided that they keep the cozie clock-face.\\


%With larger samples however, this uncontrolled situation presents an opportunity to better understand the behavioral characteristics of a user as certain patterns can be derived and interpreted. 

%If the research manager would like to have more control of the experiment they may choose to also add the "In Office / Out of Office" or the "Indoor/Outdoor" question to the list of questions asked.


%Placing participants in a controlled space and conducting surveys is a common and trusted method for human comfort research. In this methodology, users are under no control, and work from a designated co-working space at their own will, and at their own time. 


%One method commonly employed in comfort research involves placing a sample of participants in a controlled space and conducting surveys during this time. In this methodology, users are under no control. They are generally asked to work from the SDE4 building, however no direct restrictions are placed, and they are free to move as they like. (talk more about statistical significance of larger datasets and types of results infered )



\subsection{Indoor Localisation}
\label{ch:localisation}

The clock-face collects GPS data from the Fitbit, however GPS data indoors is not always reliable, and often not accessible. Only 30\% of all data points were tagged with GPS data. The team are currently investigating other methods of localization, which includes continuous logging of GPS data to infer an entry and exit of building spaces, Bluetooth based localisation from Steerpath, integration with applications focused on space use optimization, and pattern matching of wearable sensor data to indoor sensors.

% [CITE JUN]
% [CITE SpaceMatch @Clayton]

% \subsection{Problems Encountered}

% During the course of the experiment two Fitbits were lost by the users. One was lost permanently, and the other was found at a later date. There were also some issues with the collection of the environmental sensor data, resulting in only 79 matching points of the comfort feedback to the sensors, where as we expected approximately 200 matching points.





