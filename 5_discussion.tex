% !TEX root = 99_main.tex

\subsection{Large uncontrolled data vs. small controlled data}

Placing a group of participants in a controlled experimental space and conducting feedback surveys is a trusted traditional method for human comfort surveys. Giving each participant a smartwatch, and analysing the patterns of hundreds of data points per user would be more akin to modern data analytics employed in industries outside the building sector. Both methods are useful for reaching different types of conclusions. The traditional method can derive conclusions such as: \emph{4 of the 15 users felt warm at temperatures higher than 25.6 $^\circ$C}. This insight is useful in the context of the previously mentioned generalizable thermal comfort models that are traditionally created in built environment research, but have poor accuracy. On the other hand, the uncontrolled, large data method can draw conclusions such as: \emph{4 of 15 users can be categorized as a user type that prefers cooler working environments}. This paper focuses on the use of clustering to show the groups of \emph{comfort personality types}.

\subsection{In-situ benefits and limitations}
Uncontrolled experiments have minimal management overhead, which means that it can be easily scaled to larger groups by purchasing more devices. Furthermore, the users are analysed in their natural work environments and give feedback with a simple click on their watch. This reduction of effort results in no fatigue in the number of voluntary responses given as shown in Figure \ref{fig:summary}c. While users generally work from their office, they sometimes work from home, or at a local cafe. This presents a limitation in the context of traditional small controlled data. Large enough data must be obtained to filter out these scenarios and interpret meaningful results.

\subsection{Indoor Localisation}
\label{ch:localisation}

The clock-face collects GPS data from the Fitbit, however GPS data indoors is not always reliable, and often not accessible. Only 30\% of all data points were tagged with GPS data. The team is currently investigating other methods such as Bluetooth based localisation from Steerpath, and pattern matching of wearable sensor data to indoor sensors.