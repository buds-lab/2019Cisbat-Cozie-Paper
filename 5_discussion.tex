% !TEX root = 99_main.tex

\subsection{Large Uncontrolled Data vs. Small Controlled Data}

Let us take a step back and observe how market analysis was conducted. Traditionally, super markets would call individual households and ask for their feedback about what products they would like to have in stock. In modern times, applications such as facebook mine preference data from users based on what posts they "liked", and use this to generate targeted advertising for each individual.

One can draw these parallels to human comfort surveying. Placing a group of participants in a controlled experimental space and conducting feedback surveys is a trusted method for human comfort surveys. Giving each participant a smart watch, and analysing the patterns of hundreds of data points per user would be more akin to modern data analytics employed in industries outside the building sector. 



While they both work, the types of conclusions that can be derived are different. The traditional method can derive conclusions such as "4 of the 20 users felt slightly warm at temperatures higher than 25.6C $^\circ$". Whereas the uncontrolled, large data method can draw conclusions such as "4 of 20 users can be catergorised as a user type that prefers cooler working environments". To further add to this, the uncontrolled method has minimal management overhead, and can be scaled by purchasing more devices, thus providing an even richer data set along the user axis. 

\subsection{In-situ benefits and limitations}

The cozie watch face enables users to be analysed in-situ. By this we mean that the users work in their natural work environment, and give feedback with minimal effort. As introduced in Section \ref{ch:introduction} this allows the experiments to be conducted and scaled with minimal management overhead; the users don't feel like they are in an experimental setting as they are in their normal office; and finally there is no observable survey fatigue. Figure \ref{fig:responseRate} details the number of responses, including self motivated responses, per day. There is no there is no observable decrease in the feedback given.\\

The limitation however is the control of the users. Users work from their designated co-working space at their own will, and at their own time. They may decide to work from home, or are at lunch in an outdoor restaurant when giving feedback. This presents a limitation in the context of traditional small controlled datasets. With larger samples however, this uncontrolled situation presents an opportunity to better understand the behavioral characteristics of a user as certain patterns can be derived and interpreted. If the research manager would like to have more control of the experiment they may choose to also add the "In Office / Out of Office" or the "Indoor/Outdoor" question to the list of questions asked.


%Placing participants in a controlled space and conducting surveys is a common and trusted method for human comfort research. In this methodology, users are under no control, and work from a designated co-working space at their own will, and at their own time. 


%One method commonly employed in comfort research involves placing a sample of participants in a controlled space and conducting surveys during this time. In this methodology, users are under no control. They are generally asked to work from the SDE4 building, however no direct restrictions are placed, and they are free to move as they like. (talk more about statistical significance of larger datasets and types of results infered )



\subsection{Indoor Localisation}

The clock-face collects GPS data from the fitbit, however GPS data indoors is not always reliable, and often is not accessible. Other indoor localisation techniques such as pattern matching of noise data to indoor sensors [CITE JUN] are proven methods that may improve the results. Alternatively, the research institution may consider investing in indoor localisation technologies such as Steerpath which has recently been installed in buildings at the School of Design and Environment. 



\subsection{Problems Encountered}

During the course of the experiment two fitbits were lost by the users. Fortunately, one was found again, however two weeks of data was lost during this process. The other fitbit was not found. 





