% !TEX root = 99_main.tex

First trial runs of the cozie application for occupant comfort data collection have proven successful. Within just four weeks 1460 data points of thermal comfort were obtained from the 15 test participants, with minimal administrative overhead. This rich data set provides new opportunities in analysing occupant comfort behaviour through data driven methods. Within this paper, we have demonstrated how the data can be manipulated and clustered to group people into various comfort profiles. In our case there were 4 distinct groups, which can be recommended spaces that better suit their comfort profile. The data can also be clustered via time to display building defects, or anomalies in occupant behaviour. Finally, we demonstrate how the app can be combined with wearable environmental sensors to cross-reference a users preference to the environment that they were in. The open data repository for this paper can be found on Github\footnote{\url{https://github.com/buds-lab/cisbat-cozie-paper}}.

Next steps in this research involve using cozie for the exploration of occupant clustering and spatial recommendation. We will explore elements of sound, light, and thermal comfort to determine whether spatial recommendation can serve as an alternative to individualised adaptive buildings control. Furthermore, the development of a \emph{strap-pack}, a smartwatch environmental sensor that can be adapted to the watch strap is underway. 