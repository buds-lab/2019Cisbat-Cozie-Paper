% !TEX root = 99_main.tex

An experiment was deployed as part of \emph{Project Coolbit}\footnote{\url{http://www.projectcoolbit.com}}, an international effort that focusses on the use of smartwatches for human comfort analysis \cite{nazarian2019geophysics}. Fifteen participants residing in Singapore were recruited for the experiment and were equipped with Fitbit Versa or Ionic watches. Most of the participants were working at the National University of Singapore (NUS) at several flexible workspaces around campus.  The \emph{cozie} clock-face was set to request thermal preference (prefer warmer, prefer cooler, comfy), and the set to request feedback at the hours of 9:00, 11:00, 13:00, 15:00, and 17:00. 

The watch was further complemented with Internet-of-Things (IoT) connected on-body and environmental sensors. The on-body sensor consists of a temperature and light sensor from \emph{mbient-labs}\footnote{\url{https://mbientlab.com/}} that had been modified to fit the watch strap with a custom 3D printed case. An off-body sensor measuring temperature and humidity was attached to the participant's bag. The sensors communicate via Bluetooth to Raspberry-Pi gateways that had been positioned throughout the working space. Data from the cozie clock-face and the environmental sensors were aggregated in an Influx cloud time-series database, which served as a platform for data acquisition and fault detection. 
